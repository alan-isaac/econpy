\documentclass{article}
\usepackage{amsmath,amsfonts,amssymb}
\usepackage{graphicx}
\usepackage{booktabs}
\usepackage[margin=1in]{geometry}
\usepackage{moreverb}
\usepackage{natbib}  %formats references
\usepackage{setspace} 
%fpryor1@swarthmore.edu
%P.Pestieau@ulg.ac.be
\begin{document}
\author{Isaac; Caswell}
\title{Background on Pestieau 1984}
\maketitle

Here we give some background on the \citet{pestieau-1984-oep} model.


\section{Theoretical Model}

The model is basically a 2-period overlapping generations model with a single production sector.

An individual of generation t lives 2 periods, $t$ and $t+1$ (p.408).
Works only first period.

First period (p.408):
born,
receive bequest $b$,
work for real wage $w$ (equal to MPN) and with ability $a$ earn $wa$,
consume $c_{1}$,

Note that income is exogenous.

Period 1 knowledge (used to pick $c_{1}$):
own inheritance ($b$),
income ($wa)$, $r$.

Note that knowledge of $r$ should imply forward looking expectations,
given his assumed timing.

(but use r@t-1 for simulation!!!) p. 412

Tentative consumer problem:
\begin{equation}
\max_{c_{1},c_{2},x} u(c_{1}, c_{2}, x)
\end{equation}
subject to
\begin{equation}
c_{1} + (c_{2} + n b_{2})/(1+r) = w a + b
\end{equation}
and
\begin{equation}
x = w_{h}a_{h}+ b_{2}
\end{equation}
Here $x$ is the ``expected'' per-heir income during the heir's working life.%
\footnote{%
This is the proxy for per-heir ``welfare'',
so the utility function includes altruism toward heirs.
Equal division of the total bequest among heirs is assumed.
} % see Shorrocks 1979, Becker 1981
%

NOTE: in simulation, there is \emph{household} utility maximization
based on \emph{household} resources (and average household ability).
See p.412.
This matters, since husband and wife may have different resources and abilities.


Problem: How to determine $x$?
(Heirs' wage and ability unkown at time of consumer decision.)

Pestieau solution (p.409): set $w_{h}=w$ and $a_{h}=a$.
So $E[a_t+1]$ should be govenened by Pestieau (3), but is not.
That is, each heir is ``expected'' to have the same income as the (parthenogenic) parent.

Problem: Functional form for utility?

Pestieau solution:
\begin{equation}
u(c_{1}, c_{2}, x) = x^{\alpha} c_{1}^{\gamma} c_{2}^{1-\alpha-\gamma}
\end{equation}
or equivalently
\begin{equation}
\tilde{u}(c_{1}, c_{2}, x) = \alpha \ln x +\gamma \ln c_{1} + (1-\alpha-\gamma) \ln c_{2}
\end{equation}
%claims Pestieau and Posen show separability result


So we up with
\begin{equation}
\max_{c_{1}, c_{2}, b_{2}}
\alpha \ln (wa+ b_{2}) 
+ \gamma \ln c_{1}
+ (1-\alpha-\gamma) \ln c_{2}
+ \lambda [ (w a + b) - c_{1} - (c_{2} + n b_{2})/(1+r)]
\end{equation}
with first order conditions
\begin{gather}
\alpha/(wa+b_{2}) = \lambda n/(1+r)
\\
\gamma/c_{1} = \lambda
\\
(1-\alpha-\gamma)/ c_{2} = \lambda / (1+r)
\end{gather}
Substituting for $\lambda$, it follows that
\begin{gather}
\alpha/(wa+b_{2}) = \gamma n/ c_{1} (1+r)
\\
(1-\alpha-\gamma)/ c_{2} = \gamma / c_{1} (1+r)  \implies  c_{2} = c_{1} (1-\alpha-\gamma)(1+r)/\gamma 
\end{gather}
which together imply
\begin{equation}
c_{2} = (wa+b_{2})n(1-\alpha-\gamma)/\alpha \implies n b_{2} = \alpha c_{2} /(1-\alpha-\gamma)  - nwa
\end{equation}
We can therefore use the budget constraint to solve for $c_{2}$.
Substituting in the constraint for $c_{1}$ and $b_{2}$ yields
\begin{equation}
(w a + b) = \gamma c_{2}/(1-\alpha-\gamma)(1+r) + c_{2}/(1+r) + \alpha c_{2}/(1-\alpha-\gamma)(1+r) - nwa/(1+r) 
\end{equation}
so we solve for
\begin{equation}
c_{2} = (1-\alpha-\gamma)[b + wa(1+r+n)]
\end{equation}

Ooops.  Try again.
Alternative derivation:\\
Substituting for $\lambda$, it follows that
\begin{gather}
\alpha/(wa+b_{2}) = \gamma n/ c_{1} (1+r) \implies b_{2} = c_{1} (1+r) \alpha /\gamma n - wa
\\
(1-\alpha-\gamma)/ c_{2} = \gamma / c_{1} (1+r)  \implies  c_{2} = c_{1}(1+r) (1-\alpha-\gamma)/\gamma 
\end{gather}
We can therefore use the budget constraint to solve for $c_{1}$.
Note that
\begin{equation}
c_{2} + n b_{2} = c_{1}(1+r)(1-\gamma)/\gamma - nwa
\end{equation}
Substituting in the constraint for $c_{2} + n b_{2}$ yields
\begin{equation}
\begin{split}
(w a + b)
&= c_{1} + [c_{1}(1+r)(1-\gamma)/\gamma - nwa]/(1+r)
\\
&= c_{1}/\gamma - nwa/(1+r)
\end{split}
\end{equation}
so we solve for
\begin{equation}
c_{1} = \gamma[b + wa(1+r+n)/(1+r)] \implies 
(1+r)c_{1} = \gamma(1+r)(b + wa) + \gamma wan
\end{equation}
Recalling that we expressed $b_{2}$ and $c_{2}$ in terms of $c_{1}$, we therefore have
\begin{gather}
b_{2} =  [b(1+r) + wa(1+r+n)] \alpha /n - wa \implies nb_{2} = \alpha(1+r)(b+wa) - (1-\alpha)wan
\\
c_{2} = [b(1+r) + wa(1+r+n)] (1-\alpha-\gamma) = (1-\alpha-\gamma)(1+r)(b+wa) + (1-\alpha-\gamma)wan
\end{gather}
This confirms Pestieau (p.409, fn. 15), except for the capital stock measure,
which apparently assumes a single aggregate consumer.  More safely:
\begin{equation}
\begin{split}
s
&= b + aw - c_{1}
\\
&= b + aw - \gamma[b + wa(1+r+n)/(1+r)]
\\
&= (1-\gamma)(b+wa) - wa \gamma n/(1+r)
\end{split}
\end{equation}
So more correctly (right??)  %chk
\begin{equation}
K_{t+1} = N_{t} s_{t}
\end{equation}
where only the young are saving.

% Ioannides, Yannis M.
%``On the Distribution of Wealth and Intergenerational Transfers," with R. Sato, Journal of Labor Economics, 5, 3, 1987, 366-385. 
%``Heritability of Ability, Intergenerational Transfers and The Distribution of Wealth," International Economic Review, 27, 3, October 1986,  611-623.
%``Taxation and the Distribution of Income and Wealth," with R. Sato, in D. Biehl, K. Roskamp and W. Stolper, eds., Public Finance and Economic Growth, Proceedings, 37th Congress of the International Institute of Public Finance, Tokyo, September 1981, Wayne State University Press 1983, 367-386. 

Review this:
$E[r_t] = r_{t-1}$ in deriving (Pestieau 11',  p.409)
Knows number of kids he'll have!!
BUT: uncertain of kids abilities.
(Expects own ability reproduced; not RE!)

Second period:
retired,
bears $n$ heirs,
leaves $b_{2}$ to each heir at \emph{beginning} of second period,
for total bequest of $b \cdot n$.
(This is effectively intervivos transfers;
constrast simulation: end of period?)


\section{Simulation Model}

Simulation results:

Pestieau considers class marriage and random marriage, but only report class marriage results?

\begin{verbatimtab}
Parameters:
	Initialize economy with 100 *unmarried* individuals (p.413)
	run simulation for 30 periods/generations (p.413)

Simulation parameters in tables p.413 ff
	Beta:			0.5; 0.6; 0.7	[regression to the mean]
	z: 				N(0,0.15)		[Random term]
	Initial Dist:	N(0,0.15)		[Abiliy distribution]
	u-fn:	alpha	0.7				[propensity to leave bequests]
			gamma	0.2				[weight on c_1t]
	Fertility cases:
		1. Every couple has 2 children
		2. each has 1,2,3 children with prob. 0.2, 0.6, 0.2 respectively
		3. # children: Poisson dist. with mean = 2
		4. Wealthy have relatively less children (3 wealth groups): 
			# children determined by Poission dist. 
			Poor: 		mean 2.3
			In-Between: mean 2.0
			Rich:		mean 1.7
		5. Poor have relatively less children 
			# children determined by Poission dist.
			Poor:		mean 1.7
			In-Between:	mean 2.0
			Rich:		mean 2.3
\end{verbatimtab}

\begin{verbatimtab}[4]
Pestieau p.412
	allow income class differences in fertility
	allow couple to be childless. (-> ability formula not used)
		bequests of childless are "pooled and distributed" (p.415)
		odd: don't they know how many kids? Then why any bequests? clarify!
	initial wealth (K) distribution:
		"arbitrary" (p.412)
		2 cases (p.413):
			"highly unequal" (reported)
			"relatively equal" (not reported)
			equilibrium soln independent of initial distribution (p.413)
	allow negative bequests!? (p.415)
		"made possible ... by a linear bequest function"
			
	production fn is CD
	full employment, competitive factor mkts
	Capital Stock
		Initial K = SIGMA(indiv wealth)
		Thereafter = previous period savings
	**household** decision making (umax)
		u-fn is CD: x^alpha c_{1}t^gamma c_{2}t^(1-alpha-gamma)
	First marrage and then consumption and bequest decisions
		decisions made using joint earnings and joint initial wealth
			*note this is different than the parthenogenic theoretical model
	use **average** parent ability in linking parents ability to children
		*note: different than theory model
	
Pestieau p.413
	measure inequality with Gini (he does inequality of bequests across recipients?)

\end{verbatimtab}

Simulation steps:
\begin{verbatimtab}
Possible simultaneity problems:
        young cd need to know w,b to supply labor -> N -> Y
                soln: labor supply inelastic
        firm income (Y) required to pay wages and rents
                soln: first produce, then pay
Timing issue:
        old rec rents and *then* leave bequest to young

Initial round:
        Given: 100 young, K0
        So it's like the young already own the K0??

Sounds like sequence is based on Pryor (1973).
(But in Pryor, the initial generation receives both wage and property income.)
        work
        get pd
        marry
        consume/save

At the beginning of the period the old own the captial stock (->no owners first period??)
Young are born
Young work
Factor payments:
        Old receive rents (but there are no old initially!?)
        Young receive wage
Then (?) the old
        consume (rest is bequests)
        die and leave bequests
Then (?) the young
        marry
        consume
        save
So at the end of the period the young own the capital stock,
which they carry to the next period as saving.

Note the odd timing for consumption

Old bear kids
Young marry (form households)
Determine w, N in labor market (*household* umax)
Produce (K pred; N) -> Y
        distribute Y to owners of factors
                pay rents to current old
                pay wages to current young
        bequests and saving (consumption is residual)
                old leave bequests
                young save


\end{verbatimtab}

Comment: class marriage seems to be the best match to the parthenogenic theoretical model.

Question: is "lifetime income" b+wa (i.e., does it include bequests)?  Assuming yes.

\section{Simulation vs. Theoretical Model}

The simulation and theoretical models overlaps substantially but have some differences.

\begin{table}[htp]
\caption{}
\label{t:differences}
\centering
\begin{tabular}{lcc}\toprule
                        & Theory                        & Simulation
\\
Model Type              & 2 period OG                   & 2 period OG
\\
Production              & single sector                 & single sector
\\ 
Agg Prodn Fn (p.408)    & $Y_{t}=F(K_{t},N_{t})$        & $Y_{t}=F(K_{t},N_{t})$
\\ 
$w_{t}$                 & $\partial F/\partial L_{t}$   &
\\
$r_{t}$ (timing! p.408) & $\partial F/\partial K_{t+1}$ &
\\
$r$ for choices         & $r_{t}$                       & $r_{t-1}$ (simplifying!)
\\ 
$w_{h}$                 & $w$                           &
\\ 
$a_{h}$                 & $a$                           &
\\
$r_{t}$ (timing! p.408) & $\partial F/\partial K_{t+1}$ &
\\
Reproduction            & Parthenogenic                 & Marriage (class or random)
\\
Bequest timing ?        & beginning of period           & end of period (p.412)? but probably not.
\\
Bequest division        & equal                         & equal (p.412)
\\
$c_{1}$ based on        &                               &
\\
$a$ based on            & parthenogenic parent's $a$    & average of 2 parent's $a$ (p.412)
\\
\bottomrule
\end{tabular}
\end{table}


  \bibliographystyle{chicago}
  \bibliography{tex/ejabbr,tex/dist}
\end{document} 
