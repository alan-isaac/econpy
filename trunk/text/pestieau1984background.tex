\documentclass{article}
\usepackage[margin=1in]{geometry}
\usepackage{amsmath,amsfonts,amssymb}
\usepackage{graphicx}
\usepackage{booktabs}
\usepackage{moreverb}
\usepackage{natbib}  %formats references
\usepackage{setspace}
\usepackage{textcomp}
\usepackage[final]{listings}
    \lstset{% general command to set parameter(s)
        language=Python,
        rangebeginprefix=\#BEGIN\ ,%note the mandatory space
        rangeendprefix=\#END\ ,%note the mandatory space
        includerangemarker=false,
        tabsize=4,
        upquote=true,
        frame=lines,
        basicstyle=\small,          % print whole listing small
        keywordstyle=\bfseries\underbar,
                                    % underlined bold black keywords
        identifierstyle=,           % nothing happens
        commentstyle=\itshape\footnotesize, % blue comments
        stringstyle=\ttfamily,      % typewriter type for strings
        showstringspaces=false     % no special string spaces
        morestring=[b]",
        breaklines=true,
        postbreak=\space\dots
        }
\newcommand{\E}{\mathcal{E}}
%fpryor1@swarthmore.edu
%P.Pestieau@ulg.ac.be
\begin{document}
\author{Isaac; Caswell}
\title{Background on Pestieau 1984}
\maketitle

Here we give some background on the \citet{pestieau-1984-oep} model.

Core question: what is the role of the family in intergenerational inequality?
Specifically, how does family structure affect inequality in income and wealth?

Once the model is parameterized,
it is roughly a first order stochastic difference equation:
\begin{equation}
(b_{t+1}, a_{t+1}) = f(b_{t}, a_{t}, \xi_{t})
\end{equation}
where
$b_{t}$ is bequest per person,
$a_{t}$ is earnings-ability per person,
and
$\xi_{t}$ is the vector of period $t$ shocks.
(Of course, this wd literally be true only if all
individuals were identical and received identical shocks,
including sibship size.)


\section{General Background}

In a theoretical model,
\citet{stiglitz-1969-e} shows primogeniture leads to long run wealth concentration.

In a simulation model,
\citet{pryor-1973-aer} considers inheritance rules, marriage rules,
and class-based differential fertility.
Wealth and income inequality respond to all.

In a path analysis model,
\citet{clague-1977-jhr} pursues similar questions to Pestieau,
but addresses \emph{human} wealth,
and does not have fertility differences \emph{within} wealth classes.


\subsection{Stylized Facts}

Pestieau proposes these stylized facts:

Earnings resources negatively correlated with sibship size:
lowers bequests per sibling,
and lowers earnings ability by affecting intellectual growth.

Fertility varies, although variance declining in developed countries.


No strong link between fertility and income, according to Pestieau.
(But more recent studies...)

\subsubsection{Sibship Size}

Earnings ability expected to depend on
parental income,
education,
and family size.
Family size influence expected to be modest.

What is ability?  For Pestieau:

- native ability (genetic)
- measured ability (add environmental factors, including family size)
- earnings ability (add education)

See \citet[Table 1]{pestieau-1984-oep} for some evidence
on whether family size matters for earnings ability.
Overall, however, any effect seems small.

Pestieau cites \citet{cigno-1983-oep},
who allows utility to depend positively on the number of children,
but this effect is not in the Pestieau model.
(Indeed, since more children must imply a lower bequest per child,
the effect for Pestieau is negative.)
Nor does Pestieau include a fertility \emph{decision} in his model,
despite allowing fertility to vary exogenously by income class!
(A pretty fundamental conflict.)
Comment: good introduction to a simple 2-period model.

\citet{lindert-1977-jhr} finds increased sibship size reduces schooling.
(New Jersey data.)
Comment: Lindert predicted that fertility declines would reduce the
pay differential between skilled and unskilled labor.



\subsubsection{Fertility Trends}

Pestieau cites data from \citet{festy-1979-ined} to show a secular fertility decline in the last 100 years in Western countries.

\subsubsection{Income and Fertility}

Pestieau cites data from \citet{un-1976-fertility} that some Western countries (such a Belgium) show a rise in fertility with rising income, although the US shows the opposite.
Pestieau supposes there is not strong relationship,
and that cultural (?) factors matter more.

\section{Theoretical Model}

For the theoretical model, Pestieau cites \citet{becker-1981-treatise}.
However, as he notes, the model is \emph{much} simpler, and puts emphasis on exogenous uncertainty about family size.

Key focus: the intergenerational distribution of wealth rests in part in the intragenerational saving decision.
(Pestieau cites the more detailed considerations of

The model is motivated as a 2-period overlapping generations model with a single production sector.
(But this will require some discussion when we reach the simulation stage!)

An individual of generation t lives 2 periods, $1t$ and $2t$ (p.408).
Works only first period.

In any period a new generation enters the economy.
Each individual (p.408):
has some ability $a$,
receive bequest $b$,
work for real wage $w$ (equal to MPN) and thus ability to earn $wa$,
consume $c_{1}$,
save,
has $n>0$ children and educates them,
chooses a beqest $n b2$.

Note: the interpretation of $n$ is a bit tricky here.
It is really the gross population growth rate.
So $0<n<1$ is a shrinking population.

\subsection{Ability}

Note that income is predetermined (because labor supply is inelastic and
the capital stock is inherited).
A child's ability is determined from parents' ability and sibship size.
\begin{equation}
a_{t+1} = \beta a_{t} + (1-\beta)\frac{\bar{n}_{t}}{n_{t}} + z_{t}
\label{Pestieau3}\tag{Pestieau 3}
\end{equation}
Here $1-\beta \in (0,1)$ is the speed of regression to the mean ability,
$n_{t}$ is the sibship size which is random with mean $\bar{n}_{t}$,
and $z_{t}$ is an individual's ability shock (mean zero; independent of other shocks).
We implement this as follows:
\lstinputlisting[linerange=compute_ability_pestieau-compute_ability_pestieau]%
{/econpy/abs/pestieau1984oep/agents.py}

Mean ability:
\begin{equation}
\E a_{t+1} = \beta \E a_{t} + (1-\beta) \E\frac{\bar{n}_{t}}{n_{t}} + \E z_{t}
\end{equation}
Since $z$ is mean zero and since $\bar{n}/n = 1 + (\bar{n}-n)/n$, we have
\begin{equation}
\E a_{t+1} = \beta \E a_{t} + (1-\beta) + (1-\beta) \E\frac{\bar{n}-n_{t}}{n_{t}}
\end{equation}

%ai: Kyle, any thoughts on the following query?

Query: Pestieau says he assumes this to be unity, but where does he use that assumption,
which appears unjustifiable (due to Jensens' inequality).
On the other hand,
we can describe the non-stochastic dynamics ($z=0, n=\bar{n}$) by:
\begin{equation}
a_{t+1} = \beta a_{t} + (1-\beta)
\end{equation}
which naturally has a steady state value of $a_{ss}=1$.

Note: existence of a steady state $a_{ss}$ means this is not a growth model.

Pestieau also describes the evolution of the variance of $a$ as
\begin{equation}
V(a_{t+1}) = \beta^{2} V(a_{t}) + (1-\beta)^{2}\bar{n}[\E \frac{1}{n_{t}} - \frac{1}{\bar{n}}] + \sigma_{z}^{2}
\tag{Pestieau 5}
\end{equation}
with associated steady state
\begin{equation}
V^{*}(a) = \frac{(1-\beta)^{2}\bar{n}[\E \frac{1}{n_{t}} - \frac{1}{\bar{n}}] + \sigma_{z}^{2}}{1-\beta^{2}}
\tag{Pestieau 6}
\end{equation}

\subsection{Utility}

In his discussion of utility maximization,
Pestieau cites \citet{diamond-1965-aer} as the source of the theoretical model.
(This requires some additional discussion: see below.)
Two key modifications:
allow bequests,
and allow individual variation in bequests, ability, and fertility.

Features in common with Diamond:

- 2 period decision making (BUT the generations actually do not overlap in the Pestieau model).
- individuals have two periods of economic life: work first period, retired second period
- single production sector: $Y_{t} = F(K_{t}, L_{t})$.
- period $t$ saving determines period $t+1$ capital stock
- factors paid marginal product

Important:
theory treatment says (p.408) bequests left at the \emph{beginning} of the second period of life.
These would not be bequests but rather intervivos transfers.
But this is what it would take for the children to own the capital stock in the second period of the parents' lives.


Period 1 knowledge (used to pick $c_{1}$ and therefore $s = b+wa-c1$):
own inheritance ($b$),
income ($wa$),
$r$,
number of children ($n$).


NOTE: in simulation, there is \emph{household} utility maximization
based on \emph{household} resources (and average household ability).
See p.412.
This matters, since husband and wife may have different resources and abilities.


Tentative consumer problem:
\begin{equation}
\max_{c_{1},c_{2},x} u(c_{1}, c_{2}, x)
\end{equation}
subject to
\begin{equation}
c_{1} + (c_{2} + n b_{2})/(1+r) = w a + b
\label{Pestieau8} \tag{Pestieau 8}
\end{equation}
and
\begin{equation}
x = w_{h}a_{h}+ b_{2}
\end{equation}
Here $x$ is the ``expected'' per-heir income during the heir's working life.%
\footnote{%
This is the proxy for per-heir ``welfare'',
so the utility function includes altruism toward heirs.
Equal division of the total bequest among heirs is assumed.
} % see Shorrocks 1979, Becker 1981
%



Problem: Functional form for utility?

Pestieau solution:
\begin{equation}\label{utility_fn}
u(c_{1}, c_{2}, x) = x^{\alpha} c_{1}^{\gamma} c_{2}^{1-\alpha-\gamma}
\end{equation}
or equivalently
\begin{equation}\label{ln_utility_fn}
\tilde{u}(c_{1}, c_{2}, x) = \alpha \ln x +\gamma \ln c_{1} + (1-\alpha-\gamma) \ln c_{2}
\end{equation}
%claims Pestieau and Posen show separability result


Note that knowledge of $r$ should imply forward looking expectations,
given P's assumed timing.
(but use r@t-1 for simulation!!!) p. 412

So we end up with
\begin{equation}\label{u-max}
\max_{c_{1}, c_{2}, b_{2}}
\alpha \ln (w_{h}a_{h}+ b_{2})
+ \gamma \ln c_{1}
+ (1-\alpha-\gamma) \ln c_{2}
+ \lambda [ (w a + b) - c_{1} - (c_{2} + n b_{2})/(1+r)]
\end{equation}
with first order conditions
\begin{gather}
\alpha/(w_{h}a_{h}+b_{2}) = \lambda n/(1+r)
\label{dL/db2}
\\
\gamma/c_{1} = \lambda
\label{dL/dc1}
\\
(1-\alpha-\gamma)/ c_{2} = \lambda / (1+r)
\label{dL/dc2}
\end{gather}



Substituting for $\lambda$, it follows that
\begin{gather}
\alpha/(w_{h}a_{h}+b_{2}) = \gamma n/ c_{1} (1+r)
\implies
b_{2} = c_{1} (1+r)\alpha/ \gamma n - w_{h}a_{h}
\\
(1-\alpha-\gamma)/ c_{2} = \gamma / c_{1} (1+r)  \implies  c_{1} = c_{2}\gamma/(1+r) (1-\alpha-\gamma)
\end{gather}
which together imply
\begin{equation}
n b_{2} = \alpha c_{2} /(1-\alpha-\gamma)  - n w_{h} a_{h}
\end{equation}
We can therefore use the budget constraint to solve for $c_{2}$.
Substituting in the constraint for $c_{1}$ and $b_{2}$ yields
\begin{equation}
(w  a + b) = \gamma c_{2}/(1-\alpha-\gamma)(1+r) + c_{2}/(1+r) + \alpha c_{2}/(1-\alpha-\gamma)(1+r) - nw_{h}a_{h}/(1+r)
\end{equation}
\begin{equation}
(1+r)(w  a + b) = c_{2}/(1-\alpha-\gamma)  - nw_{h}a_{h}
\end{equation}
\begin{equation}
c_{2} = (1-\alpha-\gamma)[(1+r)(w  a + b)   + nw_{h}a_{h}]
\end{equation}

Problem: How to determine $x=b_{2}+w_{h}a_{h}$?
(Heirs' wage and ability unknown at time of consumer decision.)
Pestieau solution (p.409): set $w_{h}=w$ and $a_{h}=a$.
So we solve for
\begin{equation}
c_{2} = (1-\alpha-\gamma)[b(1+r) + wa(1+r+n)]
\end{equation}

Comment: not RE!

Problem: children's ability unknown.
Solution (p.408 and p.409): parent assumes children will have parent's ability.
So $E[a_t+1]$ should be governed by (\ref{Pestieau3}), but is not.
That is, each heir is ``expected'' to have the same income as the (parthenogenic) parent.

Problem: children's wage unknown.
Solution (p.409): parent assumes children will face parent's wage.

Query:
why not adopt a simpler bequest motive (e.g., u depends on b instead of wa+b)?
What is P getting out of this?

(KC: In the theory model I understand that each individual makes bequest/consumption decisions independently.  I understand that there will be a singe aggregate \emph{household} consumer in the simulation model.  That is, in the simulation model, $K_{t+1}$ represents a household's contribution to the capital stock (savings) for next period; in the theory model $K_{t+1}$ is an individuals savings which is a contribution to the next period capital stock.)\newline



Alternative derivation of same result: solve for $b_{2}$ and $c_{2}$ in terms of $c_{1}$.\\
Substituting for $\lambda$ in first order conditions, it follows that
\begin{gather}
\alpha/(wa+b_{2}) = \gamma n/ c_{1} (1+r) \implies b_{2} = c_{1} (1+r) \alpha /\gamma n - wa
\\
(1-\alpha-\gamma)/ c_{2} = \gamma / c_{1} (1+r)  \implies  c_{2} = c_{1}(1+r) (1-\alpha-\gamma)/\gamma
\end{gather}
We can therefore use the budget constraint to solve for $c_{1}$.
Note that
\begin{equation}
c_{2} + n b_{2} = c_{1}(1+r)(1-\gamma)/\gamma - nwa
\end{equation}
Substituting in the constraint for $c_{2} + n b_{2}$ yields
\begin{equation}
\begin{split}
(w a + b)
&= c_{1} + [c_{1}(1+r)(1-\gamma)/\gamma - nwa]/(1+r)
\\
&= c_{1}/\gamma - nwa/(1+r)
\end{split}
\end{equation}

Solving the equation above for $c_{1}$ we get:
\begin{equation}\label{simulation c_1}
c_{1} = \gamma[b + wa + \frac{naw}{1+r}]
\end{equation}

Note that (\ref{simulation c_1}) will be utilized in the simulation model.  However, (\ref{simulation c_1}) is easily simplified to resemble Pestieau (p.409, fn. 15) as follows:

\begin{equation}\label{opt c1 a}
c_{1} = \gamma[b + wa\frac{1+r+n} {1+r}] \implies
(1+r)c_{1} = \gamma(1+r)(b + wa) + \gamma wan 
\end{equation}
Recalling that we expressed $b_{2}$ and $c_{2}$ in terms of $c_{1}$, we therefore have
\begin{gather}
b_{2} =  [b(1+r) + wa(1+r+n)] \alpha /n - wa \implies nb_{2} = \alpha(1+r)(b+wa) - (1-\alpha)wan
\\
\begin{split}
c_{2} = [b(1+r) + wa(1+r+n)] (1-\alpha-\gamma) &= (1-\alpha-\gamma)(1+r)(b+wa) + (1-\alpha-\gamma)wan
\\
&= (1-\alpha -\gamma)[b+wa+\frac{naw}{1+r}](1+r)
\\
&= (1-\alpha -\gamma)[b+wa\frac{1+r+n}{1+r}](1+r)
\end{split}
\end{gather}
This confirms Pestieau (p.409, fn. 15), except for the capital stock measure,
which apparently assumes a single aggregate consumer.\newline

More safely:
\begin{equation}
\begin{split}
s
&= b + aw - c_{1}
\\
&= b + aw - \gamma[b + wa \frac{1+r+n}{1+r}]
\\
&= (1-\gamma)(b+wa) - wa \gamma \frac{n}{1+r}
\end{split}
\end{equation}

So more correctly (right??) if all saved identically %chk
\begin{equation}
K_{t+1} = N_{t} s_{t}
\end{equation}
where only the young are saving and $N_{t}$ is the number of young.\newline

% Ioannides, Yannis M.
%``On the Distribution of Wealth and Intergenerational Transfers," with R. Sato, Journal of Labor Economics, 5, 3, 1987, 366-385.
%``Heritability of Ability, Intergenerational Transfers and The Distribution of Wealth," International Economic Review, 27, 3, October 1986,  611-623.
%``Taxation and the Distribution of Income and Wealth," with R. Sato, in D. Biehl, K. Roskamp and W. Stolper, eds., Public Finance and Economic Growth, Proceedings, 37th Congress of the International Institute of Public Finance, Tokyo, September 1981, Wayne State University Press 1983, 367-386.

Alternatively, \citet{pestieau-1984-oep} considers the case where parent's expectations of their children's ability are governed by (\ref{Pestieau3}) and savings decisions are made using the previous period interest rate, $r_{t-1}$.  Under these assumptions the optimization problem is solved piecewise with alternative rates of return on capital for the two periods all while substituting (\ref{Pestieau3}) for $a_h$.

Under the initial assumptions we found the optimal consumption solution as stated in (\ref{opt c1 a}).  Now, with the above alternative assumptions (\ref{Pestieau3}) and $r_{t-1}$ are substituted for $a_t$ and and $r_{t}$, respectively.  It then follows that optimal first period consumption from the maximization problems yields the following:
\begin{equation}\label{opt c1 b}
\begin{split}
c_1 
&= \gamma(b+w a_{t+1} \frac{1+r_{t-1}+n}{1+r_{t-1}} )
\\
&= \gamma(b+w (\beta a_t +(1-\beta)\frac{\overline{n}}{n} + \widetilde{z}) \frac{1+r_{t-1}+n}{1+r_{t-1}}  )
\end{split}
\end{equation}
Now substitute (\ref{opt c1 b}) into the maximization problem and derive the optimal second period consumption and bequest.  That is, the the second period optimization is the following:

%KC: not sure how to make the equation below with one equation number but on 2 lines

\begin{gather*}
%\begin{eqnarray*}
%\begin{equation}
\max_{c_{2}, b_{2}}\alpha \ln (w a_{t+1} + b{2t}) 
+ \gamma \ln [ \gamma(b_t+w a_{t+1} \frac{1+r_{t-1}+n}{1+r_{t-1}}) ] 
+ (1-\alpha-\gamma) \ln c_{2}\\ 
+ \lambda [ w a_{t+1} + b - \gamma(b+w a_{t+1} \frac{1+r_{t-1}+n}{1+r_{t-1}} ) - \frac{c_{2} + n b_{2}}{1+r_t}]
%\end{equation}
%\end{eqnarray*}
\end{gather*}



where $a_{t+1} = \beta a_t + (1-\beta)\frac{\overline{n_{t}}}{n_t}+\widetilde{z_t}$, which is just (\ref{Pestieau3}).   

The first order conditions are the following:

%kc: left alignment?
\begin{eqnarray}
\frac{\partial L}{\partial b_{2t}} = \frac{\alpha}{w a_{t+1} + b_{2t}} - \lambda \frac{n_t}{1+r_t} = 0\\
\frac{\partial L}{\partial c_{2t}} = \frac{1-\alpha-\gamma}{c_{2t}}-\lambda\frac{1}{1+r_t} = 0\\
\frac{\partial L}{\partial \lambda} = w a_{t+1} + b_t - \gamma(b+w a_{t+1} \frac{1+r_{t-1}+n}{1+r_{t-1}} ) - \frac{c_{2} + n b_{2}}{1+r_t} = 0
\end{eqnarray}



--------------------------\newline
kc: Come back here!!! I haven't been able to come up with anything that resembles 11' yet.\newline
--------------------------


Review this:
$E[r_t] = r_{t-1}$ in deriving (Pestieau 11',  p.409)

Second period p.408:
retired,
bears $n$ heirs,
leaves $n b_{2}$ (as $b_{2}$ to each heir) in ``bequests'' at \emph{beginning} of second period,
for total bequest of $b_{2} \cdot n$.
No parthenogenesis: each heir receives $b_{2}$.
(This is effectively intervivos transfers;
constrast simulation: end of period?)\newline

KC: Notes from the Diamond model concerning rent payments p.1130\newline
\textbf{First Period:} t\newline
indiv works, receives wage(=MPL), allocates income between two periods (no bequests) via u-max.\newline
Consumption decisions based on the ``rate of interest for one-period loans from period $t$ to period $t+1$, $r_{t+1}$.  Thus the members of the younger generation make up the supply side of the capital market."\newline
Individual consumes difference between his/her wage and quantity he lends to capital market.  (Since there are no bequests in this model we do not have guidance regarding rents on inherited wealth for period t).\newline
\noindent\textbf{Second Period:} t+1 \newline
The same indiv consumes savings plus accrued interest: $(1+r_{t+1})s_t$, where $s_t$ is the amount saved in the first period.  So here it looks like rents on $s_t$ are eared from period $t$ until $t+1$, then are paid/received at the beginning/end(?) of period $t+1$.  (It is not possible to say when in $t+1$ that rent payments are made from the text.  However, doesn't $s_t$ have to stick around long enough to serve as the capital in production in $t+1$?  That is, consumption must occur again after production.)


%ai: complete the theory section derivations


\subsubsection{Simulation Model}

Theory more closely related to simulation model. (??)

- use $r_{t-1}$ as basis of saving decision
- Pryor style "marriages" (i.e., individuals have no sex?)
- class marriage (which is best match to theory model)
- possible income class differences in fertility (following Pryor?)
- a marriage can be childless

%ai: add derivation!

For the simulation model \citet{pestieau-1984-oep} is explicit in that he uses $r_{t-1}$ in making optimal savings decisions, p. 412.  However it is implied that parents expect that their heir's have the same (average parental for the simulation model) ability.  This is in fact a different case than the two derived above.  For this problem it must be that the first period savings optimization incorporates $r_{t-1}$ in the budget constraint.  Then optimization for the second period is derived using the actual budget constraint, using $r_t$.  Analogous to (\ref{opt c1 a}), optimal consumption for the first period will then be:
\begin{equation}\label{opt c1 c}
c_t = \gamma(b_t + w_t a_t\frac{1+r_{t-1}+n_t}{1+r_{t-1}} ) 
\end{equation}
Substituting (\ref{opt c1 c}) into the maximization problem for the second period yields:

%\begin{equation}
\begin{gather*}
\max_{c_{2t}, b_{2t}} \alpha \ln (w a_{t} + b_{2t})
+ \gamma \ln [ \gamma(b_t+w a_{t} \frac{1+r_{t-1}+n}{1+r_{t-1}}) ]
+ (1-\alpha-\gamma) \ln c_{2}\\
+ \lambda [ w a_t + b - \gamma(b+w a_t \frac{1+r_{t-1}+n}{1+r_{t-1}} ) - \frac{c_{2} + n b_{2t}}{1+r_t}]
\end{gather*}
%\end{equation}

The first order conditions are the following:

%kc: left alignment?
\begin{eqnarray}
\frac{\partial L}{\partial b_{2t}} = \frac{\alpha}{w a_t + b_{2t}} - \lambda \frac{n_t}{1+r_t} = 0\\
\frac{\partial L}{\partial c_{2t}} = \frac{1-\alpha-\gamma}{c_{2t}}-\lambda\frac{1}{1+r_t} = 0\\
\frac{\partial L}{\partial \lambda} = w a_t + b_t - \gamma(b+w a_t \frac{1+r_{t-1}+n}{1+r_{t-1}} ) - \frac{c_{2} + n b_{2}}{1+r_t} = 0
\end{eqnarray}

kc: have yet to complete\newline
--------------------------------


Bequest result changes somewhat
\begin{equation}
b_{2} =
\frac{\alpha(1+r)(b+wa)}{n}
- \frac{(1-\alpha)(1-\beta)(1+r)w\bar{n}}{n(1+r_{-1})}
- \frac{(1-\alpha)w(\beta a + z)(1+r)}{1+r_{-1}}
\label{Pestieau11p} \tag{Pestieau 11'}
\end{equation}

Query: So, apparently this what P uses for the simulation?



\section{Simulation Model}

Simulation results:

Pestieau mentions class marriage and random marriage,
but only report class marriage results.

\begin{verbatimtab}
Parameters:
	Initialize economy with 100 *unmarried* individuals (p.413)
	run simulation for 30 periods/generations (p.413)

Simulation parameters in tables p.413 ff
	Beta:			0.5; 0.6; 0.7	[regression to the mean]
	z: 				N(0,0.15)		[Random term]
	Initial Dist:	N(0,0.15)		[Abiliy distribution]
	u-fn:	alpha	0.7				[propensity to leave bequests]
			gamma	0.2				[weight on c_1t]
	Fertility cases:
		1. Every couple has 2 children
		2. each has 1,2,3 children with prob. 0.2, 0.6, 0.2 respectively
		3. # children: Poisson dist. with mean = 2
		4. Wealthy have relatively less children (3 wealth groups):
			# children determined by Poission dist.
			Poor: 		mean 2.3
			In-Between: mean 2.0
			Rich:		mean 1.7
		5. Poor have relatively less children
			# children determined by Poission dist.
			Poor:		mean 1.7
			In-Between:	mean 2.0
			Rich:		mean 2.3
\end{verbatimtab}

\begin{verbatimtab}[4]
Pestieau p.412
        bequests of childless are "pooled and distributed" (p.415)
		odd: don't they know how many kids? Then why any bequests? clarify!
	initial wealth (K) distribution:
		"arbitrary" (p.412)
		2 cases (p.413):
			"highly unequal" (reported)
			"relatively equal" (not reported)
			equilibrium soln independent of initial distribution (p.413)
	allow negative bequests!? (p.415)
		"made possible ... by a linear bequest function"
			
	production fn is CD
	full employment, competitive factor mkts
	Capital Stock
		Initial K = SIGMA(indiv wealth)
		Thereafter = previous period savings
	**household** decision making (umax)
		u-fn is CD: x^alpha c_{1}t^gamma c_{2}t^(1-alpha-gamma)
	First marrage and then consumption and bequest decisions
		decisions made using joint earnings and joint initial wealth
			*note this is different than the parthenogenic theoretical model
	use **average** parent ability in linking parents ability to children
		*note: different than theory model
	
Pestieau p.413
	measure inequality with Gini (he does inequality of bequests across recipients?)

\end{verbatimtab}

Here is the exact description of the simulation steps (p.412):
\begin{quote}
The simulation starts with 100 unmarried people with an arbitrary initial distribution of wealth.
These people who are in their first period of life work and along with the existing capital stock
generate some output and income by means of a Cobb-Douglas production function.
Full employement and mean factor payment equal to marginal productivity are assumed.
Captial stock is made of previous period saving except for the first period of the simulation in
which it is just equal to initial wealth.
Marriages then take place and each couple decides on its consumption and on the level of bequest
it wants to leave to its children,
the number of which is governed by some random process.
In making this decision, husbands and wives use their joint earnings and initial wealth.
At the end of the period, parents are removed from the scence,
bequests being equally divided among their children.
As parents usually hae different abilities, we average them when using equation (3)
linking parents and children abilities.
We now have a new group of people with a given distribution of wealth and ability;
the process can start over again and be repeated many times to see which kind of
distribution is achieved in the long-run.
\end{quote}

Simulation steps (p.412):
\begin{verbatimtab}
Setup resembles Pryor:
Initial generation of "young":
        100 people who own wealth K0
they work, producing F(K,N)
they receive wages and rents (rents too? like in Pryor)
they marry
they consume c1
they consume c2
they die and leave bequests
children enter the economy with the bequests as the new capital stock
\end{verbatimtab}
\begin{verbatimtab}
Possible simultaneity problems:
        young cd need to know w,b to supply labor -> N -> Y
                soln: labor supply inelastic
        firm income (Y) required to pay wages and rents
                soln: first produce, then pay
Timing issue:
        old rec rents and *then* leave bequest to young

Initial round:
        Given: 100 young, K0
        So it's like the young already own the K0??

Sounds like sequence is based on Pryor (1973).
(But in Pryor, the initial generation receives both wage and property income,
and a generation only lives one period)
        work
        get pd
        marry
        consume/save

KC: the first period aren't the owners of capital just the young? -> initial capital = initial wealth
AI: Sort of.  We need a consistent treatment.  I think that means the young *receive* the capital
    in the first period, just like young receive "bequests" (i.e., inter vivos transfers) in other
    periods.  But this begs the question: to whom are the rents paid in the initial period?
    I see no coherent answer to this, if the old receive the rents every other period,
    and I do not see an alternative to that.  Right?

Consider period t:
Preliminary: oldold die, oldyoung bear newcohort, oldyoung become old, newcohort becomes young
Production
Factor Payments (wages to young, rents to old)
Old leave "bequests" (i.e. itervivos transfers)
young marry and save
        note: young must marry *after* w, a, and b determined, for class marriage.  Follows Pryor.

Still the oddity:
At the beginning of the period the old own the capital stock (->no owners first period??)

Factor payments:
        Old receive rents (but there are no old initially!?)
        Young receive wage
Then (?) the old
        consume (rest is bequests)
        die and leave bequests
Then (?) the young
        marry
        consume
        save
So at the end of the period the young own the capital stock,
which they carry to the next period as saving.

Note the odd timing for consumption

Old bear kids
Young marry (form households)
Determine w, N in labor market (*household* umax)
Produce (K pred; N) -> Y
        distribute Y to owners of factors
                pay rents to current old
                pay wages to current young
        bequests and saving (consumption is residual)
                old leave bequests
                young save
\end{verbatimtab}
Comment: class marriage seems to be the best match to the parthenogenic theoretical model.

Question: is ``lifetime income'' b+wa (i.e., does it include bequests)?  We assume yes.

Oddity: c1, c2, and b2 are all picked when young and not reconsidered when old!?


\newpage
\begin{table}
\caption{Sequence of One \emph{Simulation} Period in Pestieau(1984)}
\centering
\begin{tabular}{rccccc}\toprule


 {\bf Seq} & {\bf Life Cycle} & {\bf Generation $t$} & {\bf Generation $t+1$} & {\bf Economy} & {\bf Firm} \\
\midrule
         0 &         1t & Begin Economic Life &            &  $K_{1t}$ &            \\

         1 &         1t &            &            & $w_{1t}$; $r_{t}$ &            \\

         2 &         1t &       Work &            &            &            \\

         3 &         1t &            &            &  $Y_{1t}$ &            \\

         4 &         1t &            &            &            & Factor Payments \\

         5 &         1t & receive labor and property income &            &            &            \\

         6 &         1t &      Marry &            &            &            \\

         7 &         1t & Decide: $c_{1t}$, $c_{2t}$, $b_{2t}$  &            &            &            \\

         8 &         1t & Consume $c_{1t}$ &            &            &            \\

         9 &         1t &            & Born (w/ability) &            &            \\

        10 &         2t &     Retire &            &            &            \\

        11 &         2t & rent income received(?) &            &            &            \\

        12 &         2t & consume: savings$\ast (1+r_{t+1}) ??$ &            &            &            \\

        13 &         2t &    pass on &            &            &            \\

        14 &         2t & leave $b_{2t}$ & receive  $b_{1t}$ &            &            \\
\bottomrule
%\multicolumn{6}{l}{\footnotesize $\dag$ Agents Begin Economic Life with bequests and therefore own the capital Stock  \\
\end{tabular}
\end{table}




\section{Simulation vs. Theoretical Model}

The simulation and theoretical models overlaps substantially but have some differences.

\begin{table}[htp]
\caption{}
\label{t:differences}
\centering
\begin{tabular}{lcc}\toprule
                        & Theory                        & Simulation
\\
Model Type              & 2 period OG                   & 2 period OG
\\
Production              & single sector                 & single sector
\\
Agg Prodn Fn (p.408)    & $Y_{t}=F(K_{t},N_{t})$        & $Y_{t}=F(K_{t},N_{t})$
\\
$w_{t}$                 & $\partial F/\partial L_{t}$   &
\\
$r_{t}$ (timing! p.408) & $\partial F/\partial K_{t+1}$ &
\\
$r$ for choices         & $r_{t}$                       & $r_{t-1}$ (simplifying!)
\\
$w_{h}$                 & $w$                           &
\\
$a_{h}$                 & $a$                           &
\\
$r_{t}$ (timing! p.408) & $\partial F/\partial K_{t+1}$ &
\\
Reproduction            & Parthenogenic                 & Marriage (class or random)
\\
Bequest timing ?        & beginning of period           & end of period (p.412)? but probably not.
\\
Bequest division        & equal                         & equal (p.412)
\\
$c_{1}$ based on        &                               &
\\
$a$ based on            & parthenogenic parent's $a$    & average of 2 parent's $a$ (p.412)
\\
\bottomrule
\end{tabular}
\end{table}





  \bibliographystyle{chicago}
  \bibliography{tex/ejabbr,tex/dist}
\end{document}
